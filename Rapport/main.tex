\documentclass{rapportECL}
\usepackage{lipsum}
\usepackage{biblatex} %Imports biblatex package
\addbibresource{bibliotheque.bib}

\title{Rapport ECL - Template} %Titre du fichier

\begin{document}

%----------- Informations du rapport ---------

\titre{Un Langage "Pur" pour Web Assembly} %Titre du fichier .pdf
\UE{PSTL} %Nom de la UE

\enseignant{Frédéric \textsc{Peschanski}} %Nom de l'enseignant

\eleves{Lucas \textssc{Fumard} \\
	Lauryn \textsc{Pierre} \\
	Saïd Mohammad \textsc{ZUHAIR} } %Nom des élèves

%----------- Initialisation -------------------

\fairemarges %Afficher les marges
\fairepagedegarde %Créer la page de garde
\tabledematieres %Créer la table de matières

%------------ Corps du rapport ----------------
\section{Introduction}
WebAssembly, en abrégé Wasm est une petite machine portable qui fonctionne dans les navigateurs et serveurs web modernes et sur une vaste gamme de matériel divers
\cite{jesuisundev_comprendre_2020}\cite{noauthor_webassembly_2023}.
\section{Cahier des charges}
Les tâches que nous avons identifié sont les suivantes :

- Analyser le fonctionnement de WASM

- Programmer un parseur qui puisse lire le langage pur tel que défini dans l'article\cite{ullrich_counting_2020}

- Programmer un interpréteur en Rust du langage selon les sémantiques du langage pur

- Définir quelques tests unitaires couvrant les sémantiques définies dans l'article\cite{ullrich_counting_2020}

- Ajouter la gestion des instructions \verb|inc|, \verb|dec|, \verb|reset|, \verb|reuse|

- Programmer un compilateur du langage agrandi vers WASM

\section{Tâches Réalisées}

Afin de créer le langage pur pour WASM, nous avons tout d'abord lu l'article\cite{ullrich_counting_2020} et analysé la structure du langage pur à implémenter telle que définie dans la section 3 de l'article\cite{ullrich_counting_2020}.
Cette structure nous a permis d'écrire un parseur en Rust capable de créer un AST du langage.

Par la suite, nous avons implémenté les sémantiques définies dans la section 4 sur la figure 1 en programmant un interpréteur. Nous avons testé cet interpréteur en créant plusieurs tests unitaires sur les sémantique mais aussi quelques programmes simples, tel que le calcul de fibonacci.

\section{Tâches Restantes}

Il nous faut choisir un schéma de mémoire pour les objets dans la mémoire.
Il nous faut implémenter \verb|inc|, \verb|dec|, \verb|reset|, \verb|reuse|.
Il nous faut faire le compilateur en WASM

\printbibliography
\end{document}